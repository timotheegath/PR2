\documentclass[10pt,technote]{IEEEtran}
\usepackage{amssymb}
\usepackage{amsmath}
\usepackage{graphicx}

\usepackage{subcaption}

\title{Coursework 2: Representation and Distance Metrics Learning }
\author{Timothee Gathmann \textit{(\textbf{tlg15}, 01061046)}\\ and his trusty undergrad assistanr Luka Lagator\textit{ ()}}

\begin{document}

\maketitle

\section{Problem formulation}
The features $X \in \mathbb{R}^{D X N}$ are readily available, and consist of a set of samples $x_i \in \mathbb{R}^D, i = 1, 2, ..., N$, corresponding to $N$ pictures of pedestrians. Each sample is assigned a ground-truth label $g(x_i) \in \mathbb{N}$ identifying the individual on the picture. 
We can formulate our problem as a Distance Metric Learning problem.

\end{document}